\usepackage{amsmath, amssymb}
\usepackage{comment}
\usepackage{graphicx}
%\usepackage[T1]{fontenc}
%\usepackage{ae, aecompl}
\usepackage[numbers]{natbib}

%%%%%%%%%%%%%%%%%%%%%%%%%%%%%%%%%%%%%%%%%%%%%%%%%%%%%%%%%%%%%%%%%%%%%%%%
%% math

%
% General
%

\DeclareMathOperator*{\argmin}{arg\,min}
\DeclareMathOperator*{\argmax}{arg\,max}

%
% Sets
%

\newcommand{\card} [1]{{\lvert #1 \rvert}}
\newcommand{\range}[1]{{\left\{ 1, \dots, #1 \right\}}}

%
% Linear algebra
%

\newcommand{\mat}  [1]{{\bf #1}}
\newcommand{\norm} [1]{{\lVert #1 \rVert}}
\newcommand{\vect} [1]{{\mbox{\boldmath $#1$}}}

%
% Probability. The CTAN `proba` package doesn't seem to be that good.
% TODO ensuremath does not seem to be a good idea.
%

\newcommand{\E}    [2]{{\textrm{E} \left[ #1 \mid #2 \right]}}
\newcommand{\LL}   [2]{{\ensuremath{ L \left( #1 \,\middle|\, #2 \right) }}}
\newcommand{\LLL}  [2]{{\ensuremath{ l \left( #1 \,\middle|\, #2 \right) }}}
\newcommand{\N}    [1]{{\ensuremath{ \# \left( \textrm{ #1 } \right) }}}
\newcommand{\PO}   [1]{{\Pr \left( #1 \right)}}
\newcommand{\PP}   [2]{{\PO{ #1 \,\middle|\, #2 }}}

%%%%%%%%%%%%%%%%%%%%%%%%%%%%%%%%%%%%%%%%%%%%%%%%%%%%%%%%%%%%%%%%%%%%%%%%
%% Theoretical computer science

%
% Computational complexity
%

\DeclareMathOperator{\bigO}{O}
\providecommand{\OO}[1]{\bigO\bigl(#1\bigr)}

%%%%%%%%%%%%%%%%%%%%%%%%%%%%%%%%%%%%%%%%%%%%%%%%%%%%%%%%%%%%%%%%%%%%%%%%
%% General text

% \emph toggles italics; \textit sets italics; \em is deprecated
\newcommand{\term} [1]{{\emph{#1\/}}} 
% \emph does not provide trailing italic correction
\newcommand{\Emph} [1]{{\emph{#1\/}}}

% We don't need to do append a ~ or \ because TeX already handles .,
% correctly. TODO figure out if italics are correct here
\newcommand{\foreign}[1]{{\textit{#1}}}
\newcommand{\ea}      {{\foreign{et~al.}}}
\newcommand{\eg}      {{\foreign{e.g.}}}
\newcommand{\ie}      {{\foreign{i.e.}}}
\newcommand{\TODO} [1]{{\textbf{TODO}: \textit{#1}}}

%%%%%%%%%%%%%%%%%%%%%%%%%%%%%%%%%%%%%%%%%%%%%%%%%%%%%%%%%%%%%%%%%%%%%%%%
%% Remarks. Collected from the pervasive but untraceable `remarks.tex`.

\newif\ifremark
\long\def\remark#1{
\ifremark%
    \begingroup%
    \dimen0=\columnwidth
    \advance\dimen0 by -0.25in%
    \setbox0=\hbox{\parbox[b]{\dimen0}{\protect\em\textcolor{red}{#1}}}
    \dimen1=\ht0\advance\dimen1 by 2pt%
    \dimen2=\dp0\advance\dimen2 by 2pt%
    \vskip 0.25pt%
    \hbox to \columnwidth{%
        \vrule height\dimen1 width 3pt depth\dimen2%
        \hss\copy0\hss%
        \vrule height\dimen1 width 3pt depth\dimen2%
    }%
    \endgroup%
\fi}

\newcommand{\remarkname}[2] {
    \remark{{\bf #1}:#2}
}
